%% chapter 1

\chapter{绪论}
\section{模板概述}
与Word等所见即所得编辑工具不同,使用 \LaTeX 工具排版可以将写作与排版过程分离,写作者只需要关心文字的部分,而剩下的排版工作全部交给工具自动完成。

\section{格式要求}
正文字号宋体小四,正文行间距固定为 23 点(point,pt,Word 中译作“磅”)。

空格键和 Tab       键输入的空白字符视为“空格”。连续的若干个空白字符视为一个空。一行开头的空格忽略不计。\par
行末的回车视为一个空格;但连续两个回车,也就是空行,会将文字分段。多个空行被视为一个空行。也可以在行末使用 \verb|\par| 命令分段。



使用 \verb|%| 进行注释。在这个字符之后直到行末,所有的字符都被忽略,行末的回车也不引入空格。% 我是注释

\section{各节一级标题}
我是内容

\subsection{各节二级标题}
你是内容

\subsubsection{各节三级标题}
他是内容

\paragraph{四级标题}
内容内容

\subparagraph{五级标题}
内容内容

\section{字体字号}
宋体\quad {\heiti 黑体}

\textbf{伪粗体}

\textit{伪斜体}

\textbf{\textit{伪粗斜体}}。


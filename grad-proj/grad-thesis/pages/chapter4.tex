%% chapter 4

\chapter{其它格式}
\section{代码}
\subsection{原始代码}
朴实的代码块:

使用 verbatim 环境可以得到如下原样的输出。
\begin{verbatim}
print("Hello world!")
\end{verbatim}

使用 listings 包提供的 lstlisting 环境可以对代码进行进一步的格式化,minted 包所提供的 minted 环境还可以对代码进行高亮。更多定制功能请自行参照文档配置。

\subsection{算法描述/伪代码}
参考 \href{https://en.wikibooks.org/wiki/LaTeX/Algorithms}{Algorithms} 与 algorithm2e 文档,给出一个简单的示例,见算法 \ref{alg:alg1}。

\begin{algorithm}
  \SetAlgoLined
  \KwResult{Write here the result}
  initialization\;
  \While{While condition}{
    instructions\;
    \eIf{condition}{
      instructions1\;
    }{
      instructions3\;
    }
  }
  \caption{如何写算法}\label{alg:alg1}
\end{algorithm}

\section{绘图}
关于使用 \LaTeX{} 绘图的更多例子,请参考 \href{https://www.overleaf.com/learn/latex/Pgfplots_package}{Pgfplots package}。一般建议使用如 Photoshop、PowerPoint 等制图,再转换成 PDF 等格式插入。

\section{单位}
单位的输入请使用 siunitx 包中提供的 \verb|\si| 与 \verb|\SI| 命令。在以前,\LaTeX{} 中输入角度需要使用 \verb|$^\circ$| 的奇技淫巧,现在只需要 \verb|\ang| 命令解决问题。当然 siunitx 包中还提供了不少其他有用的命令,有需要的可以自行阅读 siunitx 文档。

\section{物理符号}
physics 宏包可以让用户更加方便、简洁地使用、输入物理符号,具体也请自行阅读 physics 文档。示例如下
\begin{equation}
  \begin{aligned}
    \int_0^\frac{\symup{\pi}}{2} \abs{\sin{x}} \dd{x} & = 2 \int_0^{\symup{\pi}} \sin{x} \dd{x} \\
                                                      & = -2 \eval{\cos{x}}_0^{\symup{\pi}}     \\
                                                      & = 4
  \end{aligned}
\end{equation}

\section{写在最后}
工具不重要,对工具的合理运用才重要。希望本模板对大家的论文写作有所帮助。


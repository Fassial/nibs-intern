%% chapter 6

\chapter{总结与展望}
  诸如病毒或细菌感染之类的免疫挑战会引起组织炎症,并对下丘脑­-垂体­-肾上腺(HPA)轴产生深远影响。其中,垂体是大脑的内分泌中心,并积极参与炎症事件的调节。在这项研究中,我们对免疫攻击过程中垂体细胞的转录反应进行了探究。

  首先,这项工作提供了不同种类垂体细胞都参与中枢神经内分泌炎症调节过程的单细胞转录层级证据。LPS和Poly I:C在炎症过程中是有效的免疫刺激,将此类免疫激活剂应用于小鼠科引起严重的免疫反应。虽然以往的研究也使用这些刺激建立了小鼠炎症模型,但是其所收集到的数据集局限于垂体组织水平,无法证明不同垂体细胞均参与到炎症调节过程中。我的合作者则建立了其单细胞转录水平的数据集,弥补了垂体单细胞测序数据在炎症状态下的空缺。本文在对该数据集分析时发现,垂体中不同种类细胞在炎症状态和非炎症状态下展现出较大的转录差异,表明不同种类垂体细胞都积极参与到中枢神经内分泌炎症的调节过程中。

  其次,这项工作揭示了不同种类垂体细胞在参与中枢神经内分泌炎症调节的过程中的转录水平差异,表明其在炎症调节过程中扮演不同的角色。本文对收集到的测序数据进行基因调控网络推断,得到每一个细胞对于各基因调控通路的AUC-score矩阵。并依据此进行了Leiden聚类,将聚类结果作为判别细胞处于健康(healthy)状态还是炎症(inflammation)状态的标准。本文对不同种类垂体细胞在两种细胞状态下的基因表达差异进行了分析,发现不同种类垂体细胞在炎症状态下表达量主要调整的基因集合并不相近。这项工作从转录水平上证明了不同细胞在参与中枢神经内分泌炎症调节的过程中扮演着不同的角色。

  此外,这项工作发现了一类在不同种类垂体细胞中统一表达的转录因子,表明其在垂体参与中枢神经内分泌炎症调节过程中的重要地位。本文在转录因子的AUC-score密度分布中,找出具备双峰分布或者重尾分布的转录因子,比如Stat、Irf和Nfkb等转录因子家族,这些转录因子大多是与免疫过程相关的。本文发现这些转录因子在不同种类垂体细胞中有统一的表达,这表明其在垂体参与中枢神经内分泌炎症调节的过程中在多条调节路径上扮演着重要角色。

  最后,本文在对转录因子AUC-score矩阵进行统计的时候,发现$TNF-\alpha$刺激组在聚类结果中呈现出与其他免疫刺激组相互分离的情况。这暗示由$TNF-\alpha$介导的免疫状态转变可能涉及不同的关键转录因子,而这些转录因子将使细胞进入一种不可逆转的状态。找到这些关键转录因子将为我们设计相应的调节剂提供重要见解,有助于研发相应的抗炎症药物。

  总之,在论文中从单细胞转录水平分析了垂体细胞在参与中枢神经内分泌炎症调节过程中的共性与差异,并为未来的中枢神经内分泌炎症调节过程研究分析了方向,作为我们的未来工作。


%% abstract

% abstract-cn
\begin{abstract}
  诸如病毒或细菌感染之类的免疫挑战会引起组织炎症,并对下丘脑-垂体-肾上腺(HPA)轴产生深远影响。其中,垂体是大脑的内分泌中心,并积极参与炎症事件的调节。然而,我们对免疫攻击过程中垂体细胞的转录反应知之甚少。

  在这项研究中,我们使用免疫刺激剂(例如脂多糖(LPS)和多肌苷酸:聚胞苷酸(poly I:C))以一系列剂量和持续时间建立了炎症小鼠模型,并在其上对来自小鼠垂体的4000多个单细胞进行了单细胞RNA测序(Smart-seq2)。

  我们在垂体腺中鉴定了6个主要细胞簇,并带有相应的标记物,这与先前的知识是一致的。在这些细胞中,短期、大剂量的LPS给药引起与免疫反应、细胞因子/驱化因子释放和反应相关的基因中mRNA水平的急剧增加。在恢复3 - 5周后,这些细胞的转录状态恢复正常,但与对照组相比,某些基因似乎被上调。对单细胞转录组轨迹的伪时间分析也证实了炎症反应的过程是随后恢复到正常状态。

  这项研究的结果扩展了我们对炎症激发过程中垂体单细胞转录反应的了解,并提供了有关炎症反应过程中HPA轴激素调节的有价值的信息。
\end{abstract}

% abstract-en
\begin{abstract*}
  Immune challenges such as viral or bacterial infections cause tissue inflammations and have profound effects on the hypothalamic-pituitary-adrenal (HPA) axis. Pituitary gland is the endocrine center of the brain and is actively involved in the regulation of inflammatory events. However, very little is known about the transcriptional response of pituitary cells during immune challenge.

  In this study, we established inflammatory mouse models using immune stimuli such as lipopolysaccharides (LPS) and polyinosinic:polycytidylic acid (poly I:C) with a series of doses and durations, and performed single-cell RNA sequencing (Smart-seq2) on over 4000 individual cells from mouse pituitary gland.

  Concordantly, we identified 6 major cell clusters (Somatotropes, Corticotropes, Melanotropes, Lactotropes, Thyrotropes, Gonadotropes) in the pituitary gland with corresponding markers, which is consistent with previous knowledge. Within these cells, short-term, high-dose LPS administration invoked dramatic increase of mRNA levels in genes related to immune response, cytokine/chemokine release and response.  The transcriptional state of these cells returns to normal after 3-5 weeks of recovery, but some genes appear to be upregulated compared to control group. Pseudo-time analysis of single-cell transcriptome trajectory also confirmed that the procedure of inflammatory response was followed by recovery to normal state.

  The results from this study extends our knowledge of the transcriptional response of pituitary single cell during inflammatory challenge, and provide valuable information regarding the hormonal regulation with HPA-axis during inflammatory responses.
\end{abstract*}


%% abstract

% abstract-cn
\begin{abstract}
  诸如病毒或细菌感染之类的免疫挑战会引起组织炎症,并对下丘脑-垂体-肾上腺(HPA)轴产生深远影响。其中,垂体是大脑的内分泌中心,并积极参与炎症事件的调节。然而,我们对免疫攻击过程中垂体细胞的转录反应知之甚少。

  在这项研究中,我使用基因组学家标注的调控子-目标基因集合训练了一个GRN推断器。依据此模型,我对实验科学家在小鼠炎症模型上收集到的垂体单细胞转录组数据,进行基因调控网络推断。对得到的基因调控网络,通过聚类等分析手段,探寻免疫攻击过程中垂体细胞的转录反应。

  我在垂体数据集中鉴定出6个主要细胞簇,并带有相应的标记物,这与先前的知识是一致的。在这些细胞中,对照组与实验组之间的转录反应都有巨大差异,不同种类垂体细胞都积极参与到炎症调节过程中。但是,不同种类垂体细胞转录变化并不相同,这便表明不同垂体细胞在炎症调节过程中的不同角色。除此之外,我还鉴定出一类在不同种类垂体细胞中共表达的基因调控子,如Stat、Irf和Nfkb等。

  这项研究的结果扩展了我们对炎症激发过程中垂体单细胞转录反应的了解,并提供了有关炎症反应过程中HPA轴激素调节的有价值的信息。
\end{abstract}

% abstract-en
\begin{abstract*}
  Immune challenges such as viral or bacterial infections cause tissue inflammations and have profound effects on the hypothalamic-pituitary-adrenal (HPA) axis. Pituitary gland is the endocrine center of the brain and is actively involved in the regulation of inflammatory events. However, very little is known about the transcriptional response of pituitary cells during immune challenge.

  In this study, I trained a GRN inference machine using the regulator-target gene set annotated by genomicists. Based on this model, I performed gene regulatory network inferences on the pituitary single-cell transcriptome data collected by experimental scientists on mouse inflammation models. For the obtained gene regulatory network, clustering and other analysis methods are used to explore the transcriptional response of pituitary cells in the process of immune attack.

  I identified 6 main cell clusters in the pituitary data set with corresponding markers, which is consistent with previous knowledge. Among these cells, the transcriptional response between the control group and the experimental group is greatly different, and different types of pituitary cells are actively involved in the process of inflammation regulation. However, the transcriptional changes of different types of pituitary cells are not the same, which indicates the different roles of different pituitary cells in the regulation of inflammation. In addition, I also identified a class of gene regulators that are co-expressed in different types of pituitary cells, such as Stat, Irf, and Nfkb.

  The results from this study extends our knowledge of the transcriptional response of pituitary single cell during inflammatory challenge, and provide valuable information regarding the hormonal regulation with HPA-axis during inflammatory responses.
\end{abstract*}


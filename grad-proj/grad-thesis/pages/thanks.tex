%% thanks

\begin{acknowledgements}
  感谢蔡朝晖教授。她是大学期间为我们代课最多的老师,在我大学的学习成长过程和毕业论文撰写过程中提供了很大的指导和帮助。大一入学常听陈子轩和原昊博提起蔡老师,经过一段时间的接触,果然对学生十分负责!感谢蔡老师大学四年来对我的信任,让我在计算机本科学习中更加自信。

  感谢北京脑科学与类脑研究中心的罗敏敏教授。他在毕业论文撰写过程中给我提供了很大指导和帮助。我很敬佩罗老师对于科学的近乎狂热般的痴迷与对学生的关心。出于对神经科学的痴迷,我在保研的时候选择放弃我所擅长的计算机系统领域,转而投入神经科学领域。在刚保研完的时候,我深知我在神经科学领域的基础十分薄弱,便向罗老师申请到其实验室做毕设。罗老师并没有排斥一个本科其他专业的学生,而是在实验室本已满员的情况下十分认真地为我安排了一个相关方向的学长来指导我,让我更多地参与到实验室的课题中来,这为我以后从事神经科学领域研究打下了坚实的基础。

  与北京大学王睿宇学长的讨论让这篇论文更加完善。我们之后在神经科学领域会持续合作。

  感谢武汉大学的陈丹教授。大三上学期,选了他的一门专业选修课——计算机体系结构。陈老师在讲课的时候不拘泥于课本内容,而是十分注重培养我们的科研阅读素养。在与陈老师的交谈中,我了解到了很多体系结构前沿的研究,比如存内计算、类脑计算等,极大地开阔了我对于计算机系统的认识。也是通过陈老师的课,我对类脑计算、脑机接口与计算神经学等领域之间的联系有了初步的认知。最终走上科研道路、决定读博并且申请到北京大学前沿交叉学科研究院的PhD,陈老师起了巨大的引导作用。

  感谢武汉大学的艾浩军教授。大三下学期,由艾老师指导,和李蕴哲、朱赫合作完成的空中手写字符迁移学习项目最终成功发表,成为我人生中的第一篇论文!当时由于疫情,我们所有的交谈都被局限在线上。但艾老师每周都会与我们进行两小时以上的进展沟通,并对我们的工作提出建设性的指导意见。投稿前艾老师帮我们反复修改,在进行线上会议之前,他多次帮助我们修改海报以及展示视频,最终成功的展示离不开他的热心帮助。我们一直保持着联系。

  感谢浙江大学的潘纲教授。虽然接触的时间不长,但他的研究态度给我留下了深刻的印象——我们的电话交流永远发生在凌晨。实验室的博士生谭显瀚和祝歆韵日后都会是优秀的类脑计算研究者!这一段实习经历也让我更加坚定了从事神经科学研究的决心。

  感谢北京脑科学与类脑研究中心的周景峰教授,也是我未来的博士导师。在罗老师实验室便一直听说周师兄读博期间的各种经历,被实验室的师兄师姐一致称赞。之前在与他的交流中,我能深刻感受到他对于神经科学独到的理解与深厚的跨学科背景。虽然接触的时间还不长,但我能感受到他完美的性格!相信我们会有愉快、高产的合作。

  感谢北京生命科学联合中心的吴思教授和北京脑科学与类脑研究中心的柳昀哲教授。在与他们的交流中,我更加坚定了从神经元层级研究schema表示与修正过程的决心。我会在博士一年级到他们的实验室进行轮转。

  感谢北京生命科学研究所的王睿宇、卢立辉、袁正巍、刘志祥、曾佳为、黎亨、左鹏、于涛、全竞、严婷等同学。我十分享受与他们的每一次交流,他们对科学的严谨态度对我产生了深远的影响。

  感谢彭鹏、朱赫、李蕴哲、范文骞、章博文、周稚璇、陈子轩、原昊博等同学。感谢院学生会的同事们、WHU-MSC的朋友们、WHU-ICRobo的队友们,感谢所有的朋友。你们给我留下了永远的美好回忆!

  感谢武汉大学和弘毅学堂。学院“宽口径、厚基础、强能力”的教育方针,为我从事交叉学科的研究打下了坚实的基础。感谢弘毅学堂石兢、方萍、李瑶、董甲庆老师和辅导员王璐。

  感谢父母和家人长期的支持和鼓励。没有你们,我不会取得今天的成绩!

  四年的时光弹指一挥间,从青涩地踏进校园,到即将本科毕业,步入博士生涯。感谢过得飞快的时间,告诉我要不断努力,永不止步!

  最后用我很喜欢的一句话结束。感谢陈立杰的这句话。“能够生在这样一个黄金时代里,我感到无比的荣幸。我梦想能够成为黄金时代浪潮中的一朵浪花,为人类的智慧添砖加瓦!”
\end{acknowledgements}

